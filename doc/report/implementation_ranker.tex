\subsection{TweetRank}\label{sec:tweetrank_implementation}
Algorithm \ref{algo:alg1} is used to compute the TweetRank score and it is an adaptation of the classic Monte Carlo complete path stopping at dangling nodes method. Observe that our implementation restricts the actions that the random walker can randomly choose to the ones that are feasible given the graph from Twitter. The graph from Twitter is just the set of relationships described in section \ref{sec:tweetrank_definition} (for instance, if a tweet does not include any hashtag, the random walker won't try to jump to a hashtag to visit the next tweet).

\begin{algorithm}
\caption{Compute TweetRank using MC complete path stopping at dagling nodes}
\label{algo:alg1}
{\fontsize{8}{8}\selectfont
\begin{algorithmic}
\REQUIRE Probabilities $\alpha, \beta, \gamma, \delta, \epsilon$ such that $\alpha+\beta+\gamma+\delta+\epsilon=1$ and $\zeta > 0$, Twitter graph $G$ and $M \in \mathbb{N}$ such that $T\times M \in O(T^2)$.
\ENSURE $\pi$, TweetRank of tweets in $T(G)$.
\STATE $V_t \leftarrow 0, \forall t \in T$
\FORALL{$t \in T(G)$}
\FOR {$m=1$ to $M$}
\STATE $ct \leftarrow t$
\STATE $stop \leftarrow False$

\WHILE {$\neg stop$}
\STATE $V_{ct} \leftarrow V_{ct} + 1$ \COMMENT{Increment visit counter of $ct$}
\STATE $r \leftarrow UniformRandomNumber(0,1)$ \COMMENT{Choose what to do next}
\STATE $nt \leftarrow Null$ \COMMENT{Next tweet in the walk unknown for now}

\IF {$r < \alpha$}
\STATE $nt \leftarrow JumpToRandomTweet(ct, G)$ \COMMENT{$\alpha$ probability}
\ENDIF

\IF {$r < \alpha + \beta$}
\STATE $nt \leftarrow JumpToRetweetOrReply(ct, G)$ \COMMENT{$\beta$ probability}
\STATE $r \leftarrow UniformRandomNumber(\alpha + \beta,1)$ \COMMENT{Try an other thing to do, if no retweets or replies}
\ENDIF

\IF {$nt = Null \wedge r < \alpha +  \beta + \gamma$}
\STATE $nt \leftarrow JumpToMentionedUserTweet(ct, G)$ \COMMENT{$\gamma$ probability}
\STATE $r \leftarrow UniformRandomNumber(\alpha + \beta + \gamma,1)$ \COMMENT{Try an other thing to do, if no mentioned users}
\ENDIF

\IF {$nt = Null \wedge r < \alpha + \beta + \gamma + \delta$}
\STATE $nt \leftarrow JumpToFollowedUserTweet(ct, G)$ \COMMENT{$\delta$ probability}
\ENDIF

\IF {$nt = Null$}
\STATE $nt \leftarrow JumpToCommonHashtagTweet(ct, G)$  \COMMENT{$\epsilon$ probability}
\ENDIF

\IF {$nt = Null$} 
\STATE $stop \leftarrow True$ \COMMENT{Executed only on daggling nodes}
\ELSE
\STATE $stop \leftarrow UniformRandomNumber(0,1) < \zeta$ \COMMENT{To ensure that the algorithm will stop}
\ENDIF

\STATE $ct \leftarrow nt$ \COMMENT{Next tweet to be visited}
\ENDWHILE
\ENDFOR
\ENDFOR
\STATE $\pi_t \leftarrow Normalize(V)$ \COMMENT{TweetRank as the normalized visit vector}
\end{algorithmic}}
\end{algorithm}

Observe that the running time of algorithm \ref{algo:alg1} is non-deterministic, since it depends on the structure of the graph itself and the $\zeta$ parameter. Assuming random access memory, and no dangling nodes in graph, the previous algorithm has an expected running time bounded by $O(|T| \times M \times \mathbb{E}[|w|])$, where $\mathbb{E}[|w|]$ is the expected length of the random walk. The expected length of the random walk is obtained from the following
equation:
\begin{equation}
\mathbb{E}[|w|] = \sum_{k=1}^{\infty} P(|w| = k) \cdot k = \sum_{k=1}^{\infty} (1 - \zeta)^{k-1} \cdot \zeta \cdot k = \frac{1}{\zeta}
\end{equation}

Given that $|T| \times M$ must be $O(|T|^ 2)$ to ensure a good approximation of TweetRank, the expected running time of algorithm \ref{algo:alg1} is:
\begin{equation}\label{eq:running_time}
O(|T| \times M \times \mathbb{E}[|w|]) = O(|T|^2 \times \frac{1}{\zeta}) = O(|T|^2)
\end{equation}

Note that equation \ref{eq:running_time} is an upper bound on the expected running time if the graph contains dangling nodes. On the other side, the assumption of random access memory might be a problem for a large index which does not fit in the main memory of a single machine. However, implementing TweetRank on a large-scale distributed system is not the approach of this work and we will maintain this assumption to keep the analysis simple.